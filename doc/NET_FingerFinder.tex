\documentclass[12pt,a4paper]{article}
\usepackage[utf8]{inputenc}
\usepackage[czech]{babel}
\usepackage[T1]{fontenc}
\usepackage{amsmath}
\usepackage{amsfonts}
\usepackage{amssymb}
\usepackage{graphicx}
\usepackage{titlesec}
\usepackage[left=2cm,right=2cm,top=2cm,bottom=2cm]{geometry}
\usepackage{indentfirst}
\usepackage{listings}
\usepackage{color}
\usepackage{multirow}

%Pravidlo pro řádkování
\renewcommand{\baselinestretch}{1.5}

%Pravidlo pro začínání kapitol na novém řádku
\let\oldsection\section
\renewcommand\section{\clearpage\oldsection}

%Formáty písem pro nadpisy (-změněno na bezpatkové \sffamily z původního \normalfont
\titleformat{\section}
{\sffamily\Large\bfseries}{\thesection}{1em}{}
\titleformat{\subsection}
{\sffamily\large\bfseries}{\thesubsection}{1em}{}
\titleformat{\subsubsection}
{\sffamily\normalsize\bfseries}{\thesubsubsection}{1em}{}

%Nastavení zvýrazňování kódu v \lslisting
\definecolor{mygreen}{rgb}{0,0.6,0}
\definecolor{mygray}{rgb}{0.5,0.5,0.5}
\lstset{commentstyle=\color{mygreen},keywordstyle=\color{blue},numberstyle=\tiny\color{mygray}}

\author{Jan Šmejkal}

\begin{document}

%-------------Úvodni strana---------------
\begin{titlepage}

\includegraphics[width=50mm]{img/FAV.jpg}
\\[160 pt]
\centerline{ \Huge \sc KIV/NET - Programování v prostředí .NET}
\centerline{ \huge \sc Semestrální práce }
\\[12 pt]
{\large \sc
\centerline{Finger Finder - analyzátor otisků prstů}
}


{
\vfill 
\parindent=0cm
\textbf{Jméno:} Štěpán Ševčík\\
\textbf{Osobní číslo:} A13B0443P\\
\textbf{E-mail:} kiwi@students.zcu.cz\\
\textbf{Datum:} {\large \today\par} %datum

}

\end{titlepage}

%------------------------------------------

%------------------Obsah-------------------
\newpage
\setcounter{page}{2}
\setcounter{tocdepth}{3}
\tableofcontents
%------------------------------------------

%--------------Text dokumentu--------------
\section{Zadání}
Desktopová aplikace v programovacím jazyce C\# umožňující analýzu a klasifikaci otisků prstů
\subsection{Další požadavky na aplikaci}

Program umožní: 
\begin{itemize}
\item načítání nenormalizovaných snímků otisků prstů z fyzického média.
\item předzpracování načtených obrázků pomocí řady automatických nebo konfigurovatelných transformací, které jej připraví na analýzu a klasifikaci
\item rozpoznání neobvyklých elementů na otisku prstu, takzvaných markantů, a klasifikaci otisku do jedné z možných kategorií
\item úpravy a doladění detekovaných hodnot
\item načítání a ukládání informací o otisku prstu pro možné porovnávání
\end{itemize}


\section{Závěr}
TBA


%------------------------------------------

\end{document}